\subsection{Przebiegi sygna��w}
\subsection{Bank filtr�w akceptacyjnych}
Po otrzymaniu wiadomo�ci na magistrali CAN, gdy ta przejdzie przez filtr akceptacyjny, wraz z wiadomo�ci� przechowywana jest informacja o filtrze, kt�ry wiadomo�� dopu�ci� do systemu. Informacja ta zapisana jest w zmiennej FMI (Filter Match Index). Numer filtra nie pokrywa si� jednak z warto�ci� przechowywan� w FMI. Podczas inicjalizacji filtr�w, przed uruchomienie kontrolera CAN, nadaje si� ka�demu filtrowi unikalny numer i przypisuje si� go do konkretnej skrzynki odbiorczej FIFO (0 lub 1). W \hyperref[tab:FMI]{Tabeli~\ref*{tab:FMI}} przedstawiono odczytane warto�ci FMI wraz z wcze�niej nadanymi numerami filtr�w.



\begin{table}[h]
\caption{Numer filtru, a warto�� FMI}\label{tab:FMI}
\begin{center}
\begin{tabular}{|c|c|c|}
  \hline 
  \cellcolor{gray!50} Numer filtru & \cellcolor{gray!50} Numer FIFO & \cellcolor{gray!50} zwr�cone FMI\\
  \hline
   0 & 0 & 0\\
  \hline
   1 & 0 & 1\\
  \hline
   2 & 0 & 2\\
  \hline
   3 & 0 & 3\\
  \hline
   4 & 0 & 4\\
  \hline
   5 & 0 & 5\\
  \hline
   6 & 0 & 6\\
  \hline \hline
   7 & 1 & 0\\
  \hline 
   8 & 1 & 1\\
  \hline
   9 & 1 & 2\\
  \hline
   10 & 1 & 3\\
  \hline
   11 & 1 & 4\\
  \hline
   12 & 1 & 5\\
  \hline
   13 & 1 & 6\\
  \hline
\end{tabular} 
\end{center}
\end{table}



Wida�, �e numer FMI zale�y od kolejno�ci przyporz�dkowania filtru do konkretnego FIFO, a nie od numeru nadanego podczas inicjalizacji. Najprostszym sposobem na omini�cie problemu z�ej numeracji, jest przypisanie wszystkich filtr�w do jednego FIFO (bez pomijania �adnego filtru). Wtedy warto�� FMI pokryje si� z numerem filtru, czyli w przypadku opisywanego projektu, z numerem w�z�a, kt�ry nades�a� wiadomo��.
