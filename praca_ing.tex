
\documentclass[english, polish, bachelor, a4paper,twoside]{ppciethesis} %praca in�ynierska w j. polskim
\usepackage{polski}
\usepackage[cp1250]{inputenc}

\usepackage[OT4]{fontenc}

\usepackage{hyperref}
\usepackage[table,xcdraw]{xcolor}
\usepackage{multirow}
\usepackage{subfig}
\usepackage{float}
\usepackage{amsfonts}
\usepackage{pdfpages}
\usepackage{listings}
\definecolor{mygreen}{rgb}{0,0.6,0}
\lstset {
basicstyle=\small,
breaklines=true,
commentstyle=\color{mygreen},
keywordstyle=\color{blue},
language=C,
linewidth=\textwidth
}
\linespread{1}

\author{Marcin Aftowicz \and Jakub Baranowski} % Your name comes here
\title{Uk�ad pomiarowy do Bolidu klasy Formu�a Student (projekt~zespo�owy)} % Note how we protect the final title phrase from breaking
\ppsupervisor{dr~in�.~Dariusz~Janiszewski} % Your supervisor comes here.
\ppyear{2015}  % Year of final submission %(not graduation!)
%\hyphenation {STM32F4-Dis-co-ve-ry} 

\begin{document}
\bibliographystyle{plplain}

% Front matter starts here
\frontmatter\pagestyle{empty}%
\maketitle
%\cleardoublepage%
\Large

\thispagestyle{empty}\vspace*{\fill}
\cleardoublepage


\noindent
Podzi�kowania:\\ \\
\noindent
W tym miejscu mo�na wstawi� podzi�kowania.\\
Imi� i Nazwisko autora 1
\\ \\
\noindent
Lub nie wstawia� �adnych.\\
Imi� i Nazwisko autora 2
\\
\thispagestyle{empty}\vspace*{\fill}
\cleardoublepage

% Table of contents.
\pagenumbering{Roman}\pagestyle{ppfcmthesis}%
\tableofcontents* \cleardoublepage%
\setcounter{tocdepth}{3} %bookmarks counter = toc counter +1
\hypersetup{
linkcolor={blue!70!black},
citecolor={blue!70!black},
urlcolor={blue!70!black}
}
\begin{abstract}
Niniejsza praca in�ynierska stanowi dokumentacj� projektu realizowanego przez grup� elektryczn� w zespole PUT Motorsport w ramach konkursu \textit{Formula Student}. Jako studenci IV roku Automatyki i Robotyki Wydzia�u Elektrycznego Politechniki Pozna�skiej oraz cz�onkowie Ko�a Naukowego SENSOR zadeklarowali�my ch�� uczestnictwa w konkursie i przy��czenie si� do zespo�u.
\end{abstract}

\cleardoublepage
\mainmatter
\chapter{Wst�p oraz motywacja} \label{ch:wstep}
%=============================================================
\section{Struktura pracy} \label{sec:struktura}
Jak wygl�da rozk�ad rozdzia��w i co w nich opisano.
\section{PUT Motorsport} \label{sec:motorsport}
\textit{Formula Student (FS)} to najbardziej presti�owy, europejski konkurs w dziedzinie Motorsport, prowadzony przez \textit{Institution of Mechanical Engineers}. Wspierany przez przemys�, konkurs ma na celu inspirowanie i rozwijanie przedsi�biorczo�ci i innowacyjno�ci u m�odych in�ynier�w. Uczelnie z ca�ego �wiata maj� za zadanie zaprojektowa� i zbudowa� w pe�ni funkcjonalny samoch�d wy�cigowy, kt�ry uko�czy statyczne i dynamiczne konkurencje testuj�ce zar�wno wiedz� student�w, jak i wydajno�� pojazdu.\\
Zesp� wy�cigowy Politechniki Pozna�skiej - PUT Motorsport to dru�yna sk�adaj�ca si� z 23 student�w tworz�cych 6 grup projektowych, kt�ra podj�a si� wyzwania wystartowania w konkursie i skonstruowania w�asnego bolidu. Wyr�nia si� 6 technicznych grup projektowych odpowiedzialnych za: poszycie, zawieszenie, aerodynamik�, nap�d, materia�y oraz elektronik�. Wszystkie grupy pracuj� r�wnolegle i dbaj� o zachowanie sp�jno�ci oraz kompatybilno�ci projektowanych element�w. Specyfika takiej pracy nak�ada ograniczenia i wymusza elastyczno�� rozwi�za�, tak aby mog�y one zosta� dopasowane do ca�o�ci~\cite{misc:formula}\cite{misc:motorsport}. 
\section{Analiza problemu} \label{sec:analiza}
W ramach projektu PUT Motorsport zdecydowano si� na utworzenie systemu elektronicznego, kt�ry umo�liwi analiz� pracy poszczeg�lnych podzespo��w pojazdu. We wst�pnych fazach projektu konstruktorzy nie s� w stanie sprecyzowa� swoich wymaga�, dlatego opracowywane rozwi�zanie musi by� elastyczne i uniwersalne. Istnieje szereg cech, kt�re musi spe�nia� system i szereg ogranicze�, kt�re musz� zosta� wzi�te pod uwag�. W celu monitorowania pracy podzespo��w, potrzebne jest urz�dzenie zbieraj�ce dane z rozproszonych w poje�dzie czujnik�w. Aby m�c korzysta� z tych danych nale�y je wy�wietli�, a najlepiej zarchiwizowa�. System musi by� odporny na zak��cenia, kt�re powstan� np. w chwili zap�onu. Musi by� modyfikowalny, tak aby spe�ni� mo�liwie wiele potencjalnych potrzeb konstruktor�w (przy maksymalizacji mo�liwo�ci ingerencji w jego struktur�, minimalizacja zmian w oprogramowaniu). Musi by� kompatybilny ze sterownikiem silnika, kt�ry zostanie u�yty w celu zmian charakterystyki dzia�ania silnika. Musi posiada� �atwe i wygodne w obs�udze, intuicyjne oraz estetyczne GUI.


\section{Podzia� prac} \label{sec:podzial}
Podzia� prac nad projektem by� r�wnomierny. Ka�dy element by� konsultowany mi�dzy autorami.

\chapter{Om�wienie protoko��w komunikacyjnych} \label{ch:can}
%=============================================================
\section{G��wna magistrala komunikacyjna - CAN} \label{sec:can}
Wyb�r magistrali CAN spe�nia wszystkie za�o�enia projektu zawarte w \hyperref[sec:analiza]{Sekcji~\ref*{sec:analiza}: Analiza problemu}). Jest to protok� komunikacyjny wykorzystywany przez sterowniki silnika ECU serii PE3~\cite{manual:ecu}, kt�ry zosta� wybrany przez konstruktor�w pojazdu. Jest to powszechnie stosowany standard w systemach automatyki przemys�owej i samochodowej. Charakteryzuje si� wysokim bezpiecze�stwem (odporno�ci� na b��dy) transmisji. Pozwala na przesy�anie mikrostrumieni danych, takich jak uzyskiwane dane z czujnik�w, sk�adaj�cych si� z 1 do 8 bit�w na komunikat.

\subsection{CAN w modelu ISO/OSI} \label{sec:sub:iso}
Controller Area Network to standard przemys�owej sieci transmisyjnej, stworzonej na pocz�tku lat osiemdziesi�tych przez niemieck� firm� Bosch. Jak ka�dy powszechnie stosowany protok� komunikacyjny, tak i CAN zosta� w roku 1993  zestandaryzowany i opisany przez \textit{International Standard Organisation (ISO)} na warstwach modelu ISO/OSI i przyj�ty za norm� ISO-11898~\cite{article:siecican} (\hyperref[fig:ISO/OSI]{Rysunek~\ref*{fig:ISO/OSI}}).  Standard CAN mia� obejmowa� warstwy 1.~(fizyczn�) 2.~(��cza danych) oraz 7.~(aplikacji) \cite{article:can}.

\begin  {figure} [h] 
\centering
\includegraphics[width=0.75\textwidth]{figures/CAN_ISO_OSI.JPG}
\caption{CAN w modelu ISO/OSI~\cite{article:can}}
\label{fig:ISO/OSI}
\end {figure}

W warstwie fizycznej istniej� dwie wersje protoko�u:
\begin{itemize}
\item \textit{Low Speed CAN} od 5 kb/s do 125 kb/s
\item \textit{High Speed CAN} do 1 Mb/s
\end{itemize}
Wersje te jednak nie opisuj� bezpo�rednio realizacji fizycznej transmisji sygna�u. Powsta�o wiele dokument�w, kt�re u�ci�laj� to zagadnienie. Sygna� musi by� sygna�em r�nicowym, a najcz�ciej stosowanym medium jest skr�tka dw�ch przewod�w, ekranowanych lub nie. Sygna� r�nicowy zapobiega zniekszta�ceniu sygna�u przez zak��cenia. Topologia sieci to magistrala, co oznacza, �e wszystkie elementy sieci pod��czone s� do wsp�lnej pary przewod�w. 

W nowszej wersji specyfikacji (oznaczanej CAN 2.0), kt�ra jest odpowiedzi� na rosn�ce zapotrzebowanie, warstwa ��cza danych podzielona jest na dwie cz�ci:
\begin{itemize}
\item \textit{Logical Link Control (LLC)} odpowiedzialn� za retransmisj� danych, zarz�dzanie filtrami identyfikator�w oraz sygnalizacj� przepe�nie� skrzynek odbiorczych i nadawczych.
\item \textit{Media Access Control (MAC)} odpowiedzialn� za dost�p do medium, kodowanie i enkapsulacj� danych oraz wykrywanie b��d�w transmisji.
\end{itemize}

Dost�p do medium realizowany jest poprzez wyr�nienie dw�ch stan�w magistrali: dominuj�cego i recesywnego. (Warto�ci napi�� przedstawiono w \hyperref[tab:voltage]{Tabeli~\ref*{tab:voltage}}. Standard ISO-11898 mo�e by� stosowany r�wnie� w sieciach o ni�szych pr�dko�ciach, dlatego zaprezentowano go jako uniwersalny). 

\begin{table}[h]
\begin{center}
\begin{tabular}{|l|c|c|}
\hline
\rowcolor[HTML]{C0C0C0} 
\cellcolor[HTML]{C0C0C0} & \multicolumn{2}{l|}{\cellcolor[HTML]{C0C0C0}\textbf{Stan magistrali}} \\ \cline{2-3} 
\rowcolor[HTML]{C0C0C0} 
\multirow{-2}{*}{\cellcolor[HTML]{C0C0C0}\textbf{\begin{tabular}[c]{@{}l@{}}Napi�cie na \\ magistrali\end{tabular}}} & \multicolumn{1}{l|}{\cellcolor[HTML]{C0C0C0}\textbf{recesywny}} & \multicolumn{1}{l|}{\cellcolor[HTML]{C0C0C0}\textbf{dominuj�cy}} \\ \hline \hline
CANH & 2.5 V & 3.5 V \\ \hline
CANL & 2.5 V & 1.5 V \\ \hline
\begin{tabular}[c]{@{}l@{}}dopuszczalne\\ napi�cie\\ r�nicowe\\ $U_{0}=CANH-CANL$\end{tabular} & 0 - 0.5 V & 0.9 - 2.0 V \\ \hline
\end{tabular}
\end{center}
\caption{Warto�ci napi�� na magistrali CAN}\label{tab:voltage}
\end{table}

Je�eli urz�dzenia magistrali wymusz� jednocze�nie stan recesywny i dominuj�cy, to na linii ustabilizuje si� stan dominuj�cy. System dost�pu do medium jest potocznie zwany "iloczynem na drucie" (wi�cej informacji w \hyperref[sec:sub:filtry]{Sekcji~\ref*{sec:sub:filtry}: Filtry akceptacyjne})\\
Warstwa ��cza danych cz�sto obs�ugiwana jest sprz�towo przez kontrolery magistrali CAN, kt�re spotykane s� jako integralne cz�ci niekt�rych mikrokontroler�w.\\

Istnieje bardzo wiele r�nych standard�w opartych na warstwie aplikacji. Ka�dy producent opracowuje sw�j standard. Istniej�: 
\begin{itemize} \label{itemize:can}
\item CANopen oparty na standardzie grupy CiA (CAN in Automation - standard DS 301). Bardzo popularny protok�, u�ywany w systemach wbudowanych.
\item CAN Areospace - standard wprowadzony przez NASA (National Aeronautic and Space Administration). U�ywany do systemu kontrolno-nawigacyjnego.
\item CAN Kingdom - specyfikacja warstwy aplikacji stworzona przez szwedzka firm� Kvaser AB. Daje on projektantom swobod� w tworzeniu w�asnego systemu, otwieraj�c mo�liwo�� do projektowania systemu modu�owego.
\item Device Net - szeroko stosowany w aplikacjach automatyki przemys�owej.
\item SDS - (Smart Distributed System) - specyfikacja stworzona przez firm� Honeywell, zajmuj�c� si� systemami steruj�cymi oraz kontrolno-pomiarowymi.
\item SafetyBus - standard opracowany przez grup� Safety Network International e.V. Stosowany w przemy�le transportowym, i automatyce przemys�owej.
\item SAE - standard zdefiniowany przez grup� Society of Automotive Engineers. Stosowany jest jako system komunikacji urz�dze� kontrolnych, pomiarowych w samochodach osobowych (J1850) i ci�arowych (J1939)~\cite{misc:canbus}.
\end{itemize}

Oraz wiele innych, b�d�cych wariacjami powy�szych.

\subsection{Budowa ramki CAN} \label{sec:sub:ramkacan}
Wyr�nia si� podzia� standardu CAN na dwie kolejne grupy, wewn�trz warstwy ��cza danych:
\begin{itemize}
\item \textit CAN 2.0 A - podstawow�
\item \textit CAN 2.0 B - rozszerzon�
\end{itemize}
Podzia� ten ogranicza si� do budowy ramki, a przede wszystkim do d�ugo�ci pola arbitra�u wiadomo�ci. Podstawowa wersja ramki posiada 11-bitowy identyfikator (\hyperref[fig:subfig:canstd]{Rysunek~\ref*{fig:subfig:canstd}}), natomiast rozszerzona 29-bitowy (\hyperref[fig:subfig:canext]{Rysunek~\ref*{fig:subfig:canext}}). Dobrze zaprojektowany system mo�e skutecznie ��czy� w sobie obie wersje protoko�u.

\begin{figure} [h]
\centering
%%----start of first subfigure----
	\subfloat[Standardowa ramka]{\label{fig:subfig:canstd} 
	\includegraphics[width=0.75\textwidth]{figures/CAN_frame.JPG}}
	\\
%%----start of second subfigure----
	\subfloat[Rozszerzony nag��wek]{\label{fig:subfig:canext}
	\includegraphics[width=\textwidth]{figures/CAN_ext.JPG}}
	\caption{Ramka CAN \cite{article:siecican}}
	\label{fig:canframe} %% label for entire figure
\end{figure}

Po polu arbitra�u nast�puje pole kontrolne, w kt�rym zapisana jest informacja o ilo�ci przesy�anych danych. Kod DLC (Data Length Code) to nic innego, tylko zapis binarny liczby bajt�w przes�anych w polu danych. Maksymalna liczba to 8, czyli zakres warto�ci DLC wynosi od 0b0000 do 0b1000>~\cite{manual:stm32f4}.\\
Pole danych jest opcjonalne, gdy� istniej� ramki, kt�re s� go pozbawione, jak ramka ��dania transmisji, czy ramka przepe�nienia.

\subsection{Filtry akceptacyjne} \label{sec:sub:filtry}
Ogromn� zalet� systemu opartego na protokole CAN jest obecno�� filtr�w wiadomo�ci. W sieci CAN identyfikator wiadomo�ci jest jednocze�nie jej priorytetem. Im ni�szy identyfikator, tym wy�szy priorytet. Wynika to z faktu, �e za stan logiczny 0 odpowiada bit dominuj�cy na magistrali. Dlatego nag��wek ramki zawieraj�cy identyfikator nazywany jest polem arbitra�u. W�ze�, kt�ry chce wys�a� wiadomo�� o najmniejszym identyfikatorze, uzyska dost�p do magistrali jako pierwszy. Wszystkie w�z�y monitoruj� sie�, r�wnie� w trakcie nadawania. Gdy wykryj�, �e wiadomo�� kt�r� nadaj�, nie pokrywa si� z t� na magistrali, przestaj� nadawa� i oczekuj� na koniec ramki. Wtedy ponawiaj� pr�b� nadania wiadomo�ci~\cite{article:can}\cite{article:siecican}.\\
Wa�nym spostrze�eniem oraz znacz�c� r�nic� mi�dzy protoko�em CAN, a innymi protoko�ami, jest fakt i� protok� nie posiada adres�w. Identyfikator jest powi�zany z wiadomo�ci�, a nie z urz�dzeniem. Jako, �e ka�de urz�dzenie odczytuje stan na magistrali, wysy�anie wiadomo�ci odbywa si� w trybie broadcast. W wersji podstawowej, dzi�ki 11-bitowemu identyfikatorowi istnieje 2048 r�nych ramek, w rozszerzonej ponad 500 milion�w. Nie ma potrzeby aby wszystkie ramki by�y przetwarzane przez wszystkie w�z�y magistrali~\cite{article:can}. Kontrolery CAN umo�liwiaj� filtracj� ramek na poziomie sprz�towym, bez potrzeby anga�owania jednostki centralnej. Istniej� dwa podstawowe sposoby filtracji wiadomo�ci:
\begin{itemize}
\item Tryb maskowania. Definiujemy w nim mask�, kt�ra okre�la, kt�re bity identyfikatora b�d� por�wnywane z wzorcowym identyfikatorem. Dzi�ki temu trybowi, mo�emy w �atwy spos�b zadeklarowa� zbi�r interesuj�cych nas identyfikator�w.
\item Tryb listy identyfikator�w. Tworzymy list� identyfikator�w, kt�re b�d� akceptowane przez w�ze�. Jest to wygodne rozwi�zanie w przypadku ma�ej ilo�ci po��danych wiadomo�ci.
\end{itemize}


\section{Archiwizacja danych - SD} \label{sec:sd}
\section{Archiwizacja danych - SD} \label{sec:sd}
Mimo i� dane s� na bie��co wysy�ane do zdalnego interfejsu u�ytkownika, trzeba wzi�� pod uwag� awaryjno�� takiego przesy�u oraz mo�liwo�� gubienia pakiet�w danych przy du�ych odleg�o�ciach. Potrzebny jest stabilny i szybki system zapisu zebranych danych, kt�ry b�dzie niezale�ny od bezprzewodowej komunikacji. Wybrano zapis danych na kart� SD pod��czon� bezpo�rednio do g��wnego komputera pok�adowego. Standard kart SD jest standardem opracowanym przez trzech producent�w: Toshiba, SanDisk i MEI~\cite{manual:sandisk}, kt�ry wyewoluowa� ze starszego standardu MultiMediaCard (MMC). Zar�wno budowa samej karty, po��czenia elektryczne jak i protok� s� cz�ci� specyfikacji SD Card (SDC), podzielonej na wiele mniejszych dokument�w~\cite{spec:sd}\cite{spec:sdio}. SDC oferuje zaawansowany interfejs 9 linii elektrycznych (zegarowej, komend, 4 linie danych i 3 linie zasilania), kt�ry mo�e pracowa� z maksymaln� cz�stotliwo�ci� 50~MHz~\cite{spec:sd}.(\hyperref[fig:sd]{Rysunek~\ref*{fig:sd}}).
\begin{figure} [h]
	\centering
	\begin{minipage}[c]{0.25\linewidth}
		\centering \includegraphics[width=0.75\linewidth]{figures/SD_diagram.JPG}
	\end{minipage}%
	\begin{minipage}[c]{0.75\linewidth}
		\centering \includegraphics[width=\linewidth]{figures/SD_pinout.JPG}
	\end{minipage}
	\caption{Diagram Karty SD~\cite{manual:sandisk}}
	\label{fig:sd}
\end{figure}

Z \hyperref[fig:sd]{Rysunku~\ref*{fig:sd}} wynika, �e karty SD wspieraj� dwa fizyczne protoko�y komunikacyjne: SD Bus (\hyperref[sec:sub:sdbus]{Sekcja~\ref*{sec:sub:sdbus}: SD Bus}) oraz SPI (\hyperref[sec:sub:spi]{Sekcja~\ref*{sec:sub:spi}: Serial Peripheral Interface (SPI)}).\\
Protok� komunikacyjny kart SD opiera si� na prostym systemie komend i odpowiedzi. Wszystkie komendy s� inicjowane przez mastera. Karta SD odpowiada na zapytanie ramk� odpowiedzi, po kt�rej mo�e nast�pi� przesy� danych, je�eli taka by�a komenda, lub zg�oszenie b��du. Ca�y protok� s�u�y do obs�ugi systemu plik�w zawartego na karcie. 

\subsection{FatFs} \label{sec:sub:fat}
Z perspektywy systemu plik�w ka�dy no�nik danych podzielony jest na klastry i sektory. Sektory s� zazwyczaj d�ugo�ci 512 bajt�w, natomiast klastry przyjmuj� r�ne warto�ci, w zale�no�ci od pojemno�ci dysku i rodzaju systemu plik�w. Pliki zapisywane s� w klastrach, zajmuj�c je ca�kowicie. Oznacza to, �e gdy plik jest mniejszy od pojedynczego klastra, ca�y klaster zostanie przypisany do tego pliku. System plik�w FAT opiera si� na tablicy alokacji plik�w FAT (File Allocation Table). Jest to tablica, kt�ra stanowi katalog plik�w znajduj�cych si� na danej partycji/dysku~\cite{book:paprocki}.\\ 
FatFs to biblioteka implementuj�ca system plik�w FAT dla system�w wbudowanych. Jest to pomost ��cz�cy warstw� sprz�tow� z warstw� aplikacji. Niezale�nie od platformy sprz�towej, po zdefiniowaniu podstawowych funkcji, system zadzia�a na wybranej platformie sprz�towej. Minimalna aplikacja zak�ada, �e u�ytkownik napisze funkcje odpowiedzialne za wys�anie i odbi�r wiadomo�ci oraz inicjalizacj� karty. Dok�adny opis przewidywanego dzia�ania tych funkcji dost�pny jest na g��wnej stronie, z kt�rej pobrano bibliotek�~\cite{misc:fat}. Dodatkowo na stronie podane s� �r�d�a, z kt�rych mo�na pobra� biblioteki oparte na FatFs implementuj�ce j� na wybranych platformach sprz�towych. Jedn� z takich bibliotek, autorstwa Tilen'a Majerle~\cite{lib:sd}, u�yto w projekcie.

\subsection{SD Bus} \label{sec:sub:sdbus}
Protok� SD Bus dzieli si� na dwie wersje. Wyr�nia si� wersj� 1-bitow� oraz 4-bitow�.\\
SD Bus w wersji 1-bitowej to synchroniczny, szeregowy protok� z jedn� lini� komend, jedn� danych i jedn� zegarow�.\\
SD Bus w wersji 4-bitowej r�ni si� od niego tylko szeroko�ci� linii danych, kt�rych jest 4. Przy dobrej implementacji mo�e by� czterokrotnie szybszy ni� jego ubo�sza wersja.\\
Protok� SD Bus wymaga obliczania sumy CRC, kt�ra zapobiega b��dom transmisji. W przypadku wersji 4-bitowej, CRC liczone jest dla ka�dej linii danych z osobna. SD Bus jest domy�lnym protoko�em do obs�ugi kart SD, aby prze��czy� kart� w tryb SPI nale�y podczas inicjalizacj u�y� specjalnej komendy i przekaza� odpowiedni dla niej kod CRC~\cite{spec:sd}.

\subsection{Serial Peripheral Interface (SPI)} \label{sec:sub:spi}
Interfejs SPI s�u�y do dwukierunkowej (full duplex) , synchronicznej, szeregowej komunikacji i sk�ada si� z trzech linii:
\begin{itemize}
\item MISO - Master Input Slave Output, jednokierunkowa linia danych s�u��ca do odbierania danych przez mastera.
\item MOSI - Master Output Slave Input, jednokierunkowa linia danych s�u��ca do wysy�ania danych przez mastera.
\item SCK - linia zegarowa s�u�aca synchronizacji komunikacji~\cite{book:paprocki}.
\end{itemize}
Do aktywacji wybranego uk�adu peryferyjnego s�u�y dodatkowo linia SS (Slave Select - wyb�r uk�adu podrz�dnego).\\
Jako �e podstaw� komunikacji z kartami SD est wymiana komend i danych, a SPI nie dysponuje lini� komend, wszystkie komendy i dane s� szeregowo wysy�ane po linii MOSI i odbierane na linii MISO. Tryb SPI wspiera wi�kszo�� komend u�ywanych w komunikacji z kartami SD. Implementacja tego protoko�u jest du�o �atwiejsza ni� specyficznego SD Bus, dlatego jest to popularniejsze rozwi�zanie i zdecydowanie lepiej udokumentowane. Wi�kszo�� dzisiejszych mikrokontroler�w posiada konfigurowalne peryferium SPI. W przypadku jego braku, mo�na �atwo zaimplementowa� komunikacj� na zwyk�ych wyj�ciach cyfrowych~\cite{spec:sd}.

\subsection{Direct Memory Acces (DMA)} \label{sec:sub:dma}
Bardzo wiele operacji wykonywanych na blokach danych polega tylko na ich kopiowaniu. Nie ma potrzeby anga�owa� do tego procesu rejestr�w CPU (jednostki steruj�cej). Na potrzeby kopiowania danych, bez u�ycia procesora stworzono blok DMA (Direct Memory Acces). Je�eli rozpatrywa� peryferia jako zmapowan� pami��, mo�na u�ywa� DMA do kopiowania danych z peryferi�w do blok�w pami�ci wewn�trznej lub odwrotnie. Obs�uga karty SD mo�e odbywa� si� przy u�yciu modu�u DMA, dzi�ki czemu mo�na wskaza� kontrolerowi DMA blok pami�ci, kt�ry ma zosta� skopiowany do karty, a zapis odb�dzie si� bez u�ycia procesora.


\section{Komunikacja bezprzewodowa - XBee} \label{sec:xbee}
\section{Komunikacja bezprzewodowa - XBee} \label{sec:xbee}
\subsection{Universal Asynchronous Receiver and Transmitter - UART} \label{ssec:uart}

\chapter{Realizacja projektu}
%=============================================================
\section{Model systemu} \label{sec:model}
Zdecydowano si� na stworzenie systemu wbudowanego, kt�ry spe�ni wymagania przedstawione w (\hyperref[sec:analiza]{Sekcji~\ref*{sec:analiza}: Analiza problemu}). W celu zbierania danych potrzebny jest osobny uk�ad, kt�ry b�dzie odpowiedzialny za konkretn� grup� czujnik�w, dalej zwany jednostk� pomiarow� (HUB). Pozwala to na wi�ksz� elastyczno�� podczas doboru czujnik�w oraz protoko��w komunikacyjnych. Niweluje to r�wnie� problem przesy�u sygna�u z czujnik�w na du�e odleg�o�ci do jednego centralnego urz�dzenia, zmniejszaj�c r�wnie� ilo�� przewod�w w poje�dzie. Centralne urz�dzenie jest dalej zwane g��wnym komputerem pok�adowym. Komputer pok�adowy ma za zadanie zebranie danych, wy�wietlanie ich oraz archiwizacj�. Komunikacja mi�dzy jednostkami pomiarowymi, a g��wnym komputerem pok�adowym musi by� odporna na zak��cenia, gdy� odleg�o�ci mi�dzy tymi urz�dzeniami mog� by� znaczne, a dowolno�� u�o�enia przewod�w ograniczona przez konstrukcj� pojazdu. Komunikacja musi zapewni� wszystkim jednostkom pomiarowym czas na wys�anie wiadomo�ci, a komputerowi pok�adowemu na archiwizacj� i przesy� do interfejsu u�ytkownika. Komunikacja musi by� szybka i niezawodna. Wybrano do tego celu platform� sprz�tow� z wbudowanym kontrolerem magistrali Controller Area Network (CAN), kt�ry autonomicznie wykonuje cz�� operacji, odci��aj�c jednostk� centraln�. W celu archiwizacji danych u�yto karty SD, kt�ra zapewnia mo�liwo�� obs�ugi zar�wno przez wbudowany system, jak i przez komputer osobisty. Umo�liwia r�wnie� du�e pr�dko�ci zapisu danych, a dzi�ki wbudowanemu kontrolerowi dost�pu do pami�ci (DMA), odci��a jednostk� centraln�. Do komunikacji bezprzewodowej wybrano modu� radiowy XBee, kt�ry umo�liwia komunikacj� na znacz�ce odleg�o�ci, co gra znacz�c� rol� podczas przejazd�w pojazdu na d�ugich trasach. Jest obs�ugiwany przez zintegrowany uk�ad UART.
\begin  {figure} [h] 
\centering
\includegraphics[width=0.75\textwidth]{figures/CAN_model1.JPG}
\caption{Model systemu pomiarowego}
\label{fig:model}
\end {figure}


\section{G��wny komputer pok�adowy - Motherboard}
Komputer pok�adowy sk�ada si� z dw�ch element�w: p�ytki ewaluacyjnej STM32\-F4\-Dis\-co\-ve\-ry oraz nak�adki rozszerzaj�cej jej mo�liwo�ci. G��wnym podzespo�em komputera pok�adowego jest mikrokontroler serii STM32F4.
\subsection{Mikrokontroler} \label{sec:sub:mcu}
Nazwa serii reprezentuje kolejno nazw� producenta - STMicroelectronics, wielko�� pojedynczego rejestru - 32 bity oraz wersj� rdzenia - Cortex�-M4 CPU firmy ARM \cite{article:discovery}\cite{manual:discovery}. Jest to podrodzina rdzeni zoptymalizowana pod k�tem minimalizacji ceny przy zachowaniu du�ej wydajno�ci, przeznaczona do zastosowa� konsumenckich i przemys�owych \cite{book:paprocki}. Powodem wyboru tej platformy sprz�towej jest spe�nienie przez ni� wszystkich za�o�e� projektu odno�nie jednostki steruj�cej oraz posiadanych uk�ad�w peryferyjnych \cite{manual:stm32f4d}. Uk�ad musi 
\begin{itemize}
\item by� szybki - 168 MHz,
\item by� niezawodny - dwa timery typu watchdog,
\item posiada� rozszerzenie SDIO do obs�ugi kart SD (\hyperref[ssec:sdio]{Sekcja~\ref*{ssec:sdio}: Secure Digital Card Input Output (SDIO)}),
\item posiada� magistral� CAN do komunikacji z uk�adami pomocniczymi (\hyperref[sec:can]{Sekcja~\ref*{sec:can}: G��wna magistrala komunikacyjna - CAN}),
\item posiada� uk�ad UART do komunikacji z nadajnikiem XBee (\hyperref[ssec:uart]{Sekcja~\ref*{ssec:uart}: Universal Asynchronous Receiver and Transmitter - UART})
\item by� �atwy w  programowaniu, co umo�liwi�a rozbudowana biblioteka dostarczana przez producenta,
\end{itemize}
Zastosowanie nak�adki rozszerzaj�cej funkcjonalno�� Discovery pozwala znacznie u�atwi� proces projektowania skracaj�c go znacznie.
\subsection{Obs�uga magistrali CAN}
G��wn� magistral� komunikacyjn� w systemie jest Controller Area Network opisany w \hyperref[sec:can]{Sekcji~\ref*{sec:can}: G��wna magistrala komunikacyjna - CAN}. Implementacja komunikacji przy wykorzystaniu tej magistrali by�a g��wnym celem projektu. Dzi�ki u�yciu tego protoko�u system ma mo�liwo�� odczytywania danych ze sterownika silnika ECU serii PE3, kt�ry wysy�a wiadomo�ci przy u�yciu standardu bazuj�cego na SAE J1939. Dok�adna znajomo�� standardu SAE J1939 nie jest potrzebna w celu zbudowania systemu mog�cego si� komunikowa� ze sterownikiem. Producent do��cza not� katalogow�~\cite{manual:ecu}, w kt�rej opisane s� wszystkie mo�liwe komunikaty, kt�re sterownik wysy�a. 
\subsubsection{Przestrze� adresowa CAN}
Ka�dy identyfikator w standardzie SAE J1939, u�ywanym w sterowniku ECU, oparty jest na rozszerzonej wersji standardu CAN i posiada 29 bit�w. Ka�da wiadomo��, posiadaj�ca w�asny identyfikator, niesie ze sob� 7 lub 8 bajt�w danych, czyli maksymaln� ilo�� przewidzian� przez standard CAN. Producent sterownika ECU do��cza instrukcj� s�u��c� do dekodowania ramki, w celu uzyskania pojedynczych informacji zawartych w 8-bajtowym komunikacie~\cite{manual:ecu}. Na podstawie identyfikator�w u�ywanych przez ECU zbudowano system identyfikator�w wyst�puj�cych w systemie. Pomimo i� identyfikator przypisany jest do wiadomo�ci, a nie do urz�dzenia, utworzono system filtr�w, kt�ry jednoznacznie okre�la, kt�ry w�ze� magistrali nades�a� wiadomo��. Format identyfikatora przedstawiono na \hyperref[fig:identyfikatory]{Rysunku~\ref*{fig:identyfikatory}} i jest to przyk�ad identyfikatora u�ywanego przez sterownik silnika ECU.

\begin  {figure} [h] 
\centering
\includegraphics[width=0.75\textwidth]{figures/CAN.JPG}
\caption{identyfikatory}
\label{fig:identyfikatory}
\end {figure}

Segment A odpowiedzialny jest za adres urz�dzenia i przybiera warto�ci od 0x06 do 0x13. Zakres ten umo�liwia  obs�ug� do 13 pod��czonych urz�dze� (HUB 0 - HUB 13). Jest to niewielka ilo��, bior�c pod uwag� mo�liwo�ci jakie oferuje standard CAN. Nie zak�ada si� jednak wi�kszej potrzeby. Adresowanie zosta�o tak skonstruowane, aby identyfikator ECU, kt�rego segment A wynosi 0x0C znajdowa� si� w �rodku zakresu (HUB 6). Pami�taj�c, �e identyfikator wiadomo�ci jest jednocze�nie jej priorytetem, takie roz�o�enie identyfikator�w pozwala z du�� dowolno�ci� ustawia� system priorytet�w pomi�dzy w�z�ami.\\
Segment B odpowiedzialny jest za rozr�nienie, co zawiera komunikat. W sterowniku ECU s� to warto�ci od 0x0 do 0x6. Aby zdekodowa� komunikat nadawany przez ECU, nale�y odnie�� si� do noty katalogowej~\cite{manual:ecu}. Pozosta�e w�z�y przyjmuj� w segmencie B warto�ci od 0x0 do 0xA. Ka�da wiadomo�� skojarzona jest z kana�em przetwornika anlogowo-cyfrowego, kt�ry jest cz�ci� architektury jednostki pomiarowej.\\
Adresowanie wiadomo�ci przedstawiono na \hyperref[fig:adresowanie]{Rysunku~\ref*{fig:adresowanie}}
Pozosta�e bity identyfikatora nie s� u�ywane w systemie, stanowi�c mo�liwo�� do rozszerzenia jego funkcjonalno�ci.

\begin{figure} [h]
	\centering
	\begin{minipage}[c]{0.5\linewidth}
		\centering \includegraphics[width=0.75\linewidth]{figures/CAN.JPG}
	\end{minipage}%
	\begin{minipage}[c]{0.5\linewidth}
		\centering \includegraphics[width=0.75\linewidth]{figures/HUB.JPG}
	\end{minipage}
	\caption{Adresowanie wiadomo�ci w systemie}
	\label{fig:adresowanie}
\end{figure}



\subsubsection{Transceiver CAN}
Kontroler magistrali CAN, w kt�ry wyposa�ony jest procesor, posiada dwie asynchroniczne linie danych. S� to linie Rx oraz Tx. Aby pod��czy� kontroler do magistrali, na kt�rej wyst�puj� sygna�y Can High oraz Can Low, nale�y szeregowe komunikaty binarne przekonwertowa� na sygna� r�nicowy. Konwersj� komunikatu oraz dostosowaniem napi�� zajmuje si� Transceiver CAN. Jest to uk�ad scalony, kt�ry zasilany jest napi�ciem 5V.
\subsection{Obs�uga karty SD}
\subsection{Obs�uga modu�u XBee}
\subsection{Zasilanie}
\subsection{System przerwa� (NVIC)}

\section{Rozproszone jednostki pomiarowe - HUB}
G��wne zadanie Rozproszonej Jednostki Pomiarowej (HUB) to dokonywanie pomiaru wielko�ci fizycznych mierzonych przez czujniki rozmieszczone w bolidzie. Za standard analogowego sygna�u wej�ciowego przyj�to 0-12V. \newline
Uk�ad Jednostki Pomiarowej sk�ada si� z 3 odseparowanych galwanicznie cz�ci: pomiarowej, mikrokontrolerowej i transmisji CAN.\newline
Uk�ad realizuj�cy dzia�anie Jednostki to STM32f103, zapewniaj�c peryferia pomiarowe jak i komunikacyjne.
\subsection{Mikrokontroler}
Mikrokontroler STM32f103 pochodzi z rodziny uk�ad�w o rdzeniu Cortex\texttrademark-M3, dost�pny od paru lat na rynku, sprawdza si� w rozwi�zaniach wymagaj�cych ma�ego i prostego kontrolera. Szereg peryferi�w w kt�re wyposa�ony jest ten uk�ad stawia go w kategorii uniwersalno�ci nie osi�galnej przez inne uk�ady na rynku tej klasy. Najwa�niejsze peryferia wykorzystane w Hubie to:\newline
\begin{itemize}
	\item Dwa 12 bitowe przetworniki analogowo-cyfrowe, potrafi�ce obs�u�y� do 16 kana��w
	\item Interfejs komunikacyjny CAN 2.0B
\end{itemize}
Uk�ad dostarcza tak�e 7 timer�w sprz�towych, kt�re mog� pracowa� w wielu zaawansowanych trybach. Do wykonywania pomiar�w z okre�lonym pr�bkowaniem wystarcz� podstawowe tryby dzia�ania dostarczonych timer�w.
\subsection{Zasilanie}
\subsection{Separacja sygna��w analogowych}
Dokonywanie pomiaru odbywa si� za pomoc� kana��w ADC mikrokontrolera. Przyj�ty przez nas standard napi�cia wymusza na nas sprowadzenie poziom�w napi�� sygna�u do zakresu pracy przetwornika ADC. Podczas projektowania rozpatrywano 3 rozwi�zania:\newline
\begin{itemize}
	\item Rezystancyjny dzielnik napi�cia
	\item Izolacja sygna��w analogowych przez zewn�trzny uk�ad ADC
	\item Izolacja sygna��w analogowych na uk�adzie IL300
\end{itemize}
Najprostszym rozwi�zaniem jest rezystancyjny dzielnik napi�cia jest to czw�rnik kt�ry zapewnia uzyskanie okre�lonego stosunku pomi�dzy napi�ciem wej�ciowym a wyj�ciowym\cite{book:SE}. Rozwi�zanie tego typu w �aden spos�b nie zabezpiecza przed podaniem zbyt wysokiego napi�cia oraz wymusza aby sygna� analogowy by� mierzony wzgl�dem masy Jednostki Pomiarowej. \newline
\begin  {figure} [h] 
\centering
\includegraphics[width=0.75\textwidth]{figures/ADS1100.png}
\caption{Separacja analogowa przy u�yciu uk�adu ADS1100}
\label{fig:ADS}
\end {figure}

Nast�pn� metod� separacji, kt�r� brano pod uwag� by�o wykorzystanie zewn�trznego uk�adu ADC, kt�ry przesy�a�by dane po odseparowanej magistrali danych. Zaprojektowane rozwi�zanie wida� na \hyperref[fig:ADS]{Rysunku~\ref*{fig:ADS}}. Zosta�o odrzucone, poniewa� przetwornik wymaga� zasilania o napi�ciu $5V$ co komplikowa�o uk�ad po stronie nieseparowanej. Dodatkowo takie podej�cie podra�a�o znacz�co uk�ad. Rozwi�zanie to sprawdzi�o by si� w aplikacjach w kt�rych zale�y nam na wysokiej dok�adno�ci pomiar�w bez wprowadzania przek�ama�, kt�re pojawiaj� si� przy separacji analogowej.\newline
Ostatnia opcja, kt�ra zosta�a wybrana to separacja przy u�yciu uk�adu IL300. Uk�ad IL300 posiada jedn� diod� nadawcz� i dwie diody odbiorcze. Konfiguracja taka pozwala stworzy� po stronie pierwotnej, sprz�enie przez jedn� z diod odbiorczych, steruj�ce pr�dem diody nadawczej. Kompensuje to nieliniowo�� �wiecenia diody nadawczej wzgl�dem jej pr�du. Na stronie wt�rnej uk�adu IL300 mamy drug� diod� odbiorcz�, kt�rej pr�d jest zale�ny liniowo od pr�du diody odbiorczej po stronie pierwotnej\cite{manual:IL300}.
\begin  {figure} [h] 
\centering
\includegraphics[width=0.75\textwidth]{figures/Il300.png}
\caption{Separacja analogowa przy u�yciu uk�adu IL300}
\label{fig:IL300}
\end {figure}

\subsection{Obs�uga magistrali CAN}
Zadaniem jednostki pomiarowej jest wysy�anie zebranych informacji do g��wnego komputera pok�adowego. Nie musi ona odbiera� �adnych wiadomo�ci, st�d maska r�wna jest 0x0000000 (wi�cej o maskach w \hyperref[ssec:filtry]{Sekcji~\ref*{ssec:filtry}: Filtry akceptacyjne}).\\
Realizacja fizyczna transceivera CAN przedstawiona jest na \hyperref[fig:transceiver]{Rysunku~\ref*{fig:transceiver}}.\\
Ka�da wiadomo�� odpowiada innemu kana�owi przetwornika analogowo-cyfrowego. Czas pr�bkowania definiuje si� w pliku \textit{defines.h}. Ka�dy HUB posiada sw�j unikalny adres, kt�ry zdefiniowany jest w pliku \textit{defines.h} (wi�cej o adresowaniu w \hyperref[ssec:adresowanie]{Sekcji~\ref*{ssec:adresowanie}: Przestrze� adresowa CAN} i na \hyperref[fig:adresowanie]{Rysunku~\ref*{fig:adresowanie}}).

\section{Zdalny interfejs u�ytkownika}
Zadaniem GUI jest monitorowanie magistrali CAN oraz akwizycja danych przesy�anych przez UART do programu. Program mo�e pracowa� w dw�ch trybach:\newline
\begin{itemize}
	\item Offline- Wczytywanie danych z karty SD
	\item Online- Monitorowanie magistali CAN w czasie rzeczywistym
\end{itemize}
\subsection{Komunikacja UART}
Zdecydowano si� na komunikacje $UART$ w zwi�zku z du�� liczb� modu��w bezprzewodowych obs�uguj�cych ten typ transmisji.
Obs�uga komunikacji realizowana jest przez kontrolk� $SerialPort$, kt�ra jest cz�ci� �rodowiska Visual Studio. Dostarcza ona metod pozwalaj�cych na �atw� obs�ug� portu szeregowego.
\subsection{Grphical User Interface (GUI)}
W g��wnym komputerze pomiarowym zaimplementowano zapis ramek przesy�anych przez magistrale bezpo�rednio na kart� SD. Stworzone przez nas GUI posiada opcje wczytania logu magistrali do programu oraz manipulowania danymi. \newline
\begin  {figure} [h] 
\centering
\includegraphics[width=\textwidth]{figures/GUI_Main.PNG}
\caption{G��wne okno programu}
\label{fig:GUI_Main}
\end {figure}
Tryb online polega na przesy�aniu w czasie rzeczywistym ramek pojawiaj�cych si� na magistrali. Przesy�ana jest ca�a ramka zakodowana w kodzie szesnastkowym przez magistrale UART. Po stronie programu ramka jest wczytywana do stringa. Nast�pnym krokiem jest wczytanie ramki do napisanej klasy $Frame$, kt�ra przechowuje ramki transmisyjne oraz udost�pnia akcessory do poszczeg�lnych sk�adowych. \\newline
\begin{lstlisting}[captionpos=b, belowcaptionskip=8pt, caption=Lista mo�liwych identyfikator�w CAN, label=listing:FrameMake]
 this.Orgin = Frame;
            Adres = Orgin.Substring(0, 8);
            DLC = Orgin.Substring(8, 2);
            iDLC = int.Parse(DLC,System.Globalization.NumberStyles.HexNumber) / 2;
            if (iDLC > 2)
            {
                Canal = new string[iDLC];
                Value = new double[iDLC];
                dCanal = new double[iDLC];
                FactorA = new double[iDLC];
                FactorB = new double[iDLC];
            }
            else
            {
                Canal = new string[2];
                Value = new double[2];
                dCanal = new double[2];
                FactorA = new double[2];
                FactorB = new double[2];
            }
            for (int i = 0; iDLC != i;i++ )
            {
                Canal[i] = Orgin.Substring(10+i*4, 4);
                
                dCanal[i] = int.Parse(Canal[i], System.Globalization.NumberStyles.HexNumber);
                Value[i] = dCanal[i];
                FactorA[i] = 1;
                FactorB[i]=0;
            }
\end{lstlisting}

Na \hyperref[listing:FrameMake]{Listingu~\ref*{listing:FrameMake}} widzimy odczytywanie pierwszych 4 bajt�w adresu oraz d�ugo�ci transmisji DLC. Nast�pnie jest ona parsowana z kodu szesnastkowego do zmiennej integer. Na podstawie d�ugo�ci DLC tworzymy tablice w kt�rych zostan� ulokowane przes�ane pomiary. Przyj�to, �e ka�dy kana� pomiarowy b�dzie posiada� sw�j adres, a pomiar b�dzie przesy�any w 2 bajtach danych. Klasa $Frame$ zosta�a przystosowana tak�e do obs�ugi wielu 2 bajtowych zestaw�w danych, wys�anych przez jedn� ramk�.\newline
\begin  {figure} [h] 
\centering
\includegraphics[width=\textwidth]{figures/GUI_Main.PNG}
\caption{G��wne okno programu}
\label{fig:GUI_Main}
\end {figure}
Po uruchomieniu programu mamy dost�p do rozwijanego menu w kt�rym mo�emy wybra� na jakim porcie COM ma by� prowadzony nas�uch lub mo�liwo�� wczytania danych w trybie offline przyciskiem $Load Data$, tak jak na \hyperref[fig:GUI_Main]{Rysunku~\ref*{fig:GUI_Main}}. Po pojawieniu si� ramki na magistrali lub wczytaniu danych z pliku w oknie dialogowym widzimy z jakich adres�w przychodzi�y pomiary. Na pozycji Canal widzimy pomiar szesnastkowo, kt�ry zosta� wys�any z uk�adu pomiarowego. Warto�� pozycji Value jest przeskalowana przez wsp�czynniki $a$ i $b$ kt�re domy�lnie s� ustawione na $1$ i $0$. Aplikacja oferuje mo�liwo�� nazwania sygna�u oraz napisania komentarza co pozwala �atwiej operowa� na przychodz�cych danych. Na tej karcie mo�na obserwowa� tylko ostatni� pr�bk�, kt�ra pojawi�a si� na magistrali.\newline

\begin  {figure} [h] 
\centering
\includegraphics[width=\textwidth]{figures/GUI_Chart.PNG}
\caption{Okno wykresu}
\label{fig:GUI_Chart}
\end {figure}
Dwukrotne klikni�cie na sygna� lub zaznaczenie paru sygna��w i naci�ni�cie klawisza $Enter$ powoduje otwarcie okna wykresu danego sygna�u. W oknie wykresu mo�emy obserwowa� w czasie rzeczywistym lub offline przebieg sygna�u odczytywanego z magistrali. Przy u�yciu menu rozwijanego $Chart$ mo�emy decydowa�, kt�re sygna�y chcemy obserwowa� na wykresie.\newline
\begin  {figure} [h] 
\centering
\includegraphics[width=0.5\textwidth]{figures/GUI_Menu.PNG}
\caption{Menu w oknie Chart}
\label{fig:GUI_Chart}
\end {figure}
Menu $File$ w oknie wykresy dostarcza takich opcji jak zapis wykresu do obrazu lub generowanie pliku w formacie odczytywanym przez programy kalkulacyjne.\newline
GUI dostarcza mo�liwo�� pracy na stanowiskach wielomonitorowych. Mo�emy otworzy� okno wykresu dla ka�dego kana�u przesy�anego po magistrali CAN i rozmie�ci� je w wygodny dla u�ytkownika spos�b.\newline
\begin  {figure} [h] 
\centering
\includegraphics[width=0.8\textwidth]{figures/GUI_All.PNG}
\caption{Praca z wykresami}
\label{fig:GUI_Chart}
\end {figure}

\chapter{Badania eksperymentalne} \label{ch:eksperyment}
%=============================================================
\section{Badanie magistrali CAN}
\subsection{Przebiegi sygna��w}
\subsection{Bank filtr�w akceptacyjnych}
Po otrzymaniu wiadomo�ci na magistrali CAN, gdy ta przejdzie przez filtr akceptacyjny, wraz z wiadomo�ci� przechowywana jest informacja o filtrze, kt�ry wiadomo�� dopu�ci� do systemu. Informacja ta zapisana jest w zmiennej FMI (Filter Match Index). Numer filtra nie pokrywa si� jednak z warto�ci� przechowywan� w FMI. Podczas inicjalizacji filtr�w, przed uruchomienie kontrolera CAN, nadaje si� ka�demu filtrowi unikalny numer i przypisuje si� go do konkretnej skrzynki odbiorczej FIFO (0 lub 1). W \hyperref[tab:FMI]{Tabeli~\ref*{tab:FMI}} przedstawiono odczytane warto�ci FMI wraz z wcze�niej nadanymi numerami filtr�w.



\begin{table}[h]
\caption{Numer filtru, a warto�� FMI}\label{tab:FMI}
\begin{center}
\begin{tabular}{|c|c|c|}
  \hline 
  \cellcolor{gray!50} Numer filtru & \cellcolor{gray!50} Numer FIFO & \cellcolor{gray!50} zwr�cone FMI\\
  \hline
   0 & 0 & 0\\
  \hline
   1 & 0 & 1\\
  \hline
   2 & 0 & 2\\
  \hline
   3 & 0 & 3\\
  \hline
   4 & 0 & 4\\
  \hline
   5 & 0 & 5\\
  \hline
   6 & 0 & 6\\
  \hline \hline
   7 & 1 & 0\\
  \hline 
   8 & 1 & 1\\
  \hline
   9 & 1 & 2\\
  \hline
   10 & 1 & 3\\
  \hline
   11 & 1 & 4\\
  \hline
   12 & 1 & 5\\
  \hline
   13 & 1 & 6\\
  \hline
\end{tabular} 
\end{center}
\end{table}



Wida�, �e numer FMI zale�y od kolejno�ci przyporz�dkowania filtru do konkretnego FIFO, a nie od numeru nadanego podczas inicjalizacji. Najprostszym sposobem na omini�cie problemu z�ej numeracji, jest przypisanie wszystkich filtr�w do jednego FIFO (bez pomijania �adnego filtru). Wtedy warto�� FMI pokryje si� z numerem filtru, czyli w przypadku opisywanego projektu, z numerem w�z�a, kt�ry nades�a� wiadomo��.


\section{Badanie protoko��w karty SD}
Por�wnanie protoko��w SPI z SD Bus

\chapter{Podsumowanie}
%=============================================================

% All appendices and extra material, if you have any.
\cleardoublepage\appendix%
%\input{0a-Opis_zawartosci_plyty.tex}
\chapter{Rysunki techniczne}

dupa
Twoja stara

%\cleardoublepage
\hypersetup{ linkcolor={black}}
\cleardoublepage
\listoftables % %lista tabel na osobnej stronie
\cleardoublepage
\listoffigures % %lista rysunków na osobnej stronie
\bibliography{bibliografia}
\ppcolophon
\end{document}